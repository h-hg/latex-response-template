\documentclass{article}
\usepackage[utf8]{inputenc}

% template
%% You can change this flag to `False` after finish
\def\IfDraft{True}
%% Template package
\usepackage{responseletter}
%% You can define your custom package in this file
% Using [H] to ban the float of algorithm, figure and so on.
\usepackage{float}

% Figure
\usepackage{graphicx}
\usepackage[]{subfig}

% Table
\usepackage{multirow}

% Math
\usepackage{amsmath}
\usepackage{amssymb}

% URL
\usepackage{url}

% Algorithm
%% Float of algorithm
\usepackage{algorithm}
%% Body of algorithm
\usepackage{algorithmic}
%% Custom setting for algorithmic
\renewcommand{\algorithmicrequire}{\textbf{Input:}} 
\renewcommand{\algorithmicensure}{\textbf{Output:}}

% Code highlight
\usepackage{minted}

% Autoref
\usepackage[]{hyperref} 
\hypersetup{
  colorlinks=true,
}
%% Custom refname for algorithm
\newcommand{\algorithmautorefname}{Algorithm}
%% Custom refname for subfigure
\newcommand{\subfigureautorefname}{Figure}
%% Custom refname for section
\renewcommand{\sectionautorefname}{Section}
%% Custom refname for subsection
\renewcommand{\subsectionautorefname}{Subsection}
%% Custom refname for step of algorithm
\makeatletter
\newcommand{\ALC@uniqueautorefname}{Line}
\makeatother

% border box
\usepackage{boxedminipage}
\usepackage{framed}

% Input the information of paper here
%% The title of the paper
\def\PaperTitle{Title of the paper}
%% ID of paper
\def\PaperId{{JRNL\_YEAR\_NUM}}
%% Revison number
\def\PaperRevision{1}
%% Special issue id
\def\SpecialIssueId{{SI:CONF20XX}}
%% The name of journal or conference
\def\Journal{{Journal of Something}}

%% Names of all authors 
\def\Authors{Name 1, Name 2}
%% Name of author, in the form of `[ID of affiliation]{name}`
\author[1]{Name 1}
\author[2]{Name 2}
%% Information of affiliation, in the form of `[ID of affiliation]{information}`
\affil[1]{University 1, City 1, Country 1}
\affil[2]{University 3, City 2, Country 2}

% Don't show the date
\date{}
\title{\textbf{\PaperTitle}}

% Add your bibtex file here
\addbibresource{ref.bib}

\begin{document}
  
  \maketitle
  
  % Content of the 1st page
  \noindent \textbf{Dear Editors},
  \\[2em]
  \indent We express our gratitude for the time and effort dedicated to the reviewing of our submitted manuscript.
  We worked diligently to address all the concerns raised by the referees.
  Below we provide our detailed response to their comments.
  %We have highlighted the main changes in the paper by colouring the modified text and adding a side note in which the addressed remark is referenced.
  We hope that the applied revisions are to the satisfaction of the editors.
  \\[2em]
  Kind regards,
  \begin{flushright}
    \Authors
  \end{flushright}
  \vfill
  \section*{Manuscript information}
  \begin{description}
    \item[Number:] \PaperId
    \item[Title:] ``\PaperTitle''
    \item[Authors:] \Authors
    \item[Revision:] \PaperRevision
    \item[Submitted to:] \Journal
    \item[Special issue:] \SpecialIssueId
  \end{description}
  \pagebreak
  
  {
    \hypersetup{linkcolor=black}
    \tableofcontents
    \pagebreak
  }

  \begin{Editor}[Associate Editor]
  \begin{CommentSummary}
    Hope the reviewers' comments would be useful for your research.
  \end{CommentSummary}
  
  \begin{Response}
    Thank you very much.
    We have revised the manuscript by taking reviewers’ comments and suggestions into consideration to enhance the paper quality.
    The revisions are marked in {\color{blue}blue} color in the revised manuscript.
  \end{Response}
\end{Editor}
  \begin{Reviewer}

  \begin{CommentSummary}
    The paper puts forward a new methodology for XXX.
    The concepts are clearly described and the algorithmic flows are presented carefully.
    Experiments are carried out in comparison to state-of-the-art methods on a set of benchmark functions.
  \end{CommentSummary}
  \begin{Response}
    Thank you very much for your valuable comments on our paper and work.
    We have revised the manuscript by taking your following comments and suggestions into consideration to enhance the paper quality.
    The revisions are marked in {\color{blue}blue} color in the revised manuscript. 
  \end{Response}

  \begin{ReviewerComment}
    The writing is generally good but could be further improved.
  \end{ReviewerComment}
  
  \begin{Response}
    Thank you for your valuable comment.
    We have further polished up the language presentations.
    We hope the revised paper will be more clear and accurate on expressions.
  \end{Response}

  \begin{ReviewerComment}
    The process of XXX is not clear.
  \end{ReviewerComment}
  
  \begin{Response}
    Thank you for your valuable comment. 
    In the revised paper, we have added some XXX.
    The revised description are shown in the Section XXX-X. 
  \end{Response}

  \begin{ReviewerComment}
    There are some typos and informal notations, e.g., the decision vector is a vector and thus it should be given as $\mathbf{x}$.
  \end{ReviewerComment}
  \begin{Response}
    Thank you very much for pointing out this issue.
    We have changed the notations according to your suggestion and double-checked the manuscript to avoid similar problems. 
  \end{Response}


  \begin{ReviewerComment}
    The format of the references should be unified, e.g., [8]  and [38].
  \end{ReviewerComment}

  \begin{Response}
    Thank you very much for pointing out this issue.
    We have carefully formatted all the references in the revised manuscript.
    The above three reference formats are presented as follows:

    \begin{enumerate}
      \item[\texttt{[8]}\enspace] \fullcite{kahn1962topological}

      \item[\texttt{[38]}] \fullcite{he2016deep}
    \end{enumerate}
  \end{Response}

  \Tip You can use \mintinline{latex}{\,} or \mintinline{latex}{\enspace} to align the numbering of \mintinline{latex}{\item} to the left.
  For more details, please refer to the tex files.

\end{Reviewer}

  \section{How to use}

\subsection{Basic usage}

In general, this template provides several environment to identify different reviewers and editors.
A simple example is as follows.

\begin{minted}[]{latex}
\begin{Editor}[Associate Editor]
  \begin{CommentSummary}
    A summary/general comment of associate editor.
  \end{CommentSummary}
  \begin{Response}
    Your response.
  \end{Response}
\end{Editor}
\begin{Reviewer}
  \begin{CommentSummary}
    A summary/general comment of reviewer.
  \end{CommentSummary}
  \begin{Response}
    Your response.
  \end{Response}
  \begin{ReviewerComment}
    A comment of the reviewer.
  \end{ReviewerComment}
  \begin{Response}
    Your response.
  \end{Response}
\end{Reviewer}
\end{minted}

Reviewers are automatically numbered using Arabic numerals.
Reviewer responses are numbered using Roman numerals automatically.
More details, please see the .tex file.

\subsection{Display of content}

An introduction to topological sorting will be used to introduce the arrangement of content, such as figures, algorithms and reference.

Topological sort is an algorithm for sorting a directed acyclic graph (DAG).
It arranges all the vertices $v$ in the graph $G$ into a linear sequence $L$, making the starting vertex of any edge in $G$ is arranged before its ending vertex in $L$.
From the perspective of discrete mathematics, the vertices of an edge can be regarded as a partial order, and then topological sorting can be defined as obtaining a total order of the set from the set of partial orders.
The most typical implementation of topological sorting is the Kahn algorithm~\cite{kahn1962topological}, which continuously removes the vertex with zero indegree in the graph $G$ and append the vertex into the end of the current sequence $L$.
Its pseudo code is shown in \autoref{algo:kaha}, and \autoref{fig:kaha-example} is an example of this pseudo code.

  \begin{algorithm}[H]
    \caption{KahnAlgorithm}
    \label{algo:kaha}
    \begin{algorithmic}[1]
  \REQUIRE Graph $G(\mathbb{V}, \mathbb{E})$
  \ENSURE Sequence $L$
  \STATE $L \leftarrow$ an empty sequence
  \STATE $Q \leftarrow$ the vertices whose indegree is zero
  \WHILE{$Q$ is not empty}
    \STATE $u \leftarrow$ remove the top node of $Q$
    \STATE add $u$ to $L$
    \FOR{each node $v$ with an edge $e$ from $u$ to $v$}
      \STATE remove edge $e$ from graph $G$
      \IF{indegree of $v$ is $0$}
        \STATE push $v$ to $Q$
      \ENDIF
    \ENDFOR
  \ENDWHILE
  \RETURN $L$
\end{algorithmic}

  \end{algorithm}

\begin{figure}[H]
  \centering
  \subfloat[\label{fig:kaha-example-a}]{
    \includegraphics[width=0.2\textwidth, page=1]{img/kaha-example.drawio.pdf}
  }
  \subfloat[\label{fig:kaha-example-b}]{
    \includegraphics[width=0.2\textwidth, page=2]{img/kaha-example.drawio.pdf}
  }
  \subfloat[\label{fig:kaha-example-c}]{
    \includegraphics[width=0.2\textwidth, page=3]{img/kaha-example.drawio.pdf}
  }
  \\
  \subfloat[\label{fig:kaha-example-d}]{
    \includegraphics[width=0.2\textwidth, page=4]{img/kaha-example.drawio.pdf}
  }
  \subfloat[\label{fig:kaha-example-e}]{
    \includegraphics[width=0.2\textwidth, page=5]{img/kaha-example.drawio.pdf}
  }
  \subfloat[\label{fig:kaha-example-f}]{
    \includegraphics[width=0.2\textwidth,page=6]{img/kaha-example.drawio.pdf}
  }
  \caption{
    A example of Kaha algorithm.
    In \autoref{fig:kaha-example-a}, indegree of $v_0$ and $v_1$ is zero, thus, they are pushed into the $Q$.
    In \autoref{fig:kaha-example-b}, $v_0$ is popped out, the edge starting from it is removed, and then put $v_0$ into $L$.
    In \autoref{fig:kaha-example-c}, $v_1$ is popped out and put into $L$, the edge $e_{12}$ is removed, $v_2$ is pushed into $Q$.
    In \autoref{fig:kaha-example-d}, $v_2$ is popped out and put into $L$, edges $e_{23}$ and $e_{24}$ are removed, $v_4$ is pushed into $Q$.
    In \autoref{fig:kaha-example-e}, $v_4$ is popped out and pushed into $L$, edge $e_{43}$ is removed, $v_3$ is pushed into $Q$.
    In \autoref{fig:kaha-example-f}, $v_3$ is popped out and push into $L$.
  }
  \label{fig:kaha-example} 
\end{figure}

\printbibliography[heading=none]

\subsection{Draft Mode}

There is a draft mode is this template.
The content in environment `Draft' will only be displayed when the draft mode is turned on.

\begin{minted}[]{latex}

\begin{Draft}
  This section will only be displayed if you set `IfDraft` to `True` in the `letter.tex`.
  You can set `IfDraft` to `False` after finishing the letter.
\end{Draft}

\end{minted}

\subsection{Label}

The Label command can be used to display the label.
It can be used with draft mode like follow.

\begin{minted}[]{latex}
\begin{Response}
  Your response here.
  \begin{Draft}
    \begin{flushright}
      \textbf{Assigment:} Someone who needs to reply to it\\
      \Hard \\
    \end{flushright}
  \end{Draft}
\end{Response}
\end{minted}

The effect is as follows.

\begin{framed}
\noindent \textbf{Response}~\\

Your response here.

\begin{flushright}
  \textbf{Assigment:} Someone who needs to reply to it\\
  \Hard \\
\end{flushright}
\end{framed}

The built-in labels are list as follows.
\begin{itemize}
    \item \Hard
    \item \Medium
    \item \Easy
    \item \Done
\end{itemize}

You can customize your labels by
\begin{minted}[]{latex}
\Label{color}{text}
\end{minted}


\end{document}
